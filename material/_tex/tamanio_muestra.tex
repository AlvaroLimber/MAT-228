\documentclass[11pt,openbib, letter]{article}

\oddsidemargin=1cm

% Incluir los paquetes necesarios 
\usepackage[latin1]{inputenc}%paquete principal2
\usepackage{latexsym} % Simbolos 
\usepackage{graphicx} % Inclusion de graficos. Soporte para figura 
\usepackage{hyperref} % Soporte hipertexto
\usepackage[spanish,es-noquoting]{babel}
\usepackage{subfigure}
\usepackage{epsfig}
\usepackage{graphicx}
\usepackage{rotating}
\usepackage{epstopdf}
\usepackage{ctable}
\usepackage{longtable}
\usepackage{tabularx,colortbl}
\usepackage{setspace}
\usepackage{threeparttable}
\usepackage{multirow}
\usepackage{pdflscape}
\usepackage{anysize}
%\usepackage[authoryear]{natbib}
\usepackage{breakurl}
\usepackage{url}
\usepackage{amssymb}
\usepackage{graphicx}
\usepackage[centertags]{amsmath}
\usepackage{amsthm}
\usepackage{array}
\usepackage{times}
\usepackage[left=0.7in, right=0.7in, top=0.7in, bottom=0.7in]{geometry}
\usepackage{rotating}
\usepackage{amstext}
\usepackage{pdfpages}
\usepackage{amsbsy}
\usepackage{amsopn}
\usepackage{eucal}
\usepackage{sectsty}
\usepackage{titlesec}
\usepackage[capposition=top]{floatrow}
\usepackage{pdfpages}
\usepackage{apacite}
\usepackage{tabularx,colortbl}
%\usepackage[singlespacing]{setspace}
%\usepackage[longnamesfirst,comma]{natbib}
%\usepackage{iadb}
\usepackage{authblk}
\title{{\Large \textbf{Consideraciones para el Tama�o de Muestra\\ Trabajo Social}}}
\author{Docente: Alvaro Limber Chirino Gutierrez}

\date{Mayo, 2019}
\begin{document}
%\maketitle
\begin{flushright}
\textbf{Docente: Alvaro Chirino Gutierrez, Estad�stica Social. I-2021}
\end{flushright}
\section{Formula General para el tama�o de  muestra}
\begin{equation}
n_{final}=\frac{n_{1}*(Deff)}{(1-\hat{TNR})}
\end{equation}
Con:\\
$n_{final}:$ Tama�o de muestra final (En unidades elementales)\\
$Deff:$ Efecto de Dise�o (Se recomienda un $Deff=2$ si no se lo puede calcular. Si se realiza un muestreo aleatorio simple su valor es 1).\\
$\hat{TNR}:$ Tasa de No Respuesta Esperada, es la proporci�n que se estima perder por no respuesta o boletas mal llenadas.\\
$n_1:$ Tama�o de muestra para un muestreo de tipo aleatoria simple para una poblaci�n finita.

\begin{equation}
n_{1}=\frac{n_0}{1+\frac{n_0}{N}}
\end{equation}
Con:\\
$N:$ Tama�o de la Poblaci�n (En unidades elementales)\\
$n_0:$ Tama�o de muestra para un muestreo de tipo aleatorio simple para poblaciones infinitas\\

\subsection{Error Absoluto}
\begin{itemize}
\item Indicadores de tipo promedio ($\bar{Y}$)
\begin{equation}
n_0=\frac{k^2*S^2_y}{e^2}
\end{equation}
\item Indicadores de tipo proporci�n ($P$)
\begin{equation}
n_0=\frac{k^2*N*P*Q}{e^2*(N-1)}
\end{equation}
\end{itemize}

\subsection{Error Relativo}
\begin{itemize}
\item Indicadores de tipo promedio ($\bar{Y}$)
\begin{equation}
n_0=\frac{k^2*CV^2_y}{e^2_r}
\end{equation}
\item Indicadores de tipo proporci�n ($P$)
\begin{equation}
n_0=\frac{k^2*Q*N}{e^2_r*(N-1)*P}
\end{equation}
\end{itemize}

Con:\\
$k:$ Coeficiente de confiabilidad (al 95\% de confiabilidad $k=1.96$)\\
$S_y:$ Desviaci�n estandar de la variable $Y$ (Buscar en alguna fuente similar)\\
$e:$ Margen de error permisible Absoluto (Definir en base a cuanto se quiere alejar en t�rminos absolutos del verdadero par�metro)\\
$e_r:$ Margen de error permisible Relativo (Ideal $e_r==0.05$, mayor a 0.15 es cuestionable)\\
$P:$ Proporci�n de una categor�a de inter�s (Para una muestra con error relativo, mientras mas extra�a es la categoria a explorar mas grande es la muestra requerida. Para muestras con error absoluto el m�ximo se alcanza en $P=0.5$)\\
$Q=1-P$\\
$CV_y:$ Es el Coeficiente de Variaci�n de la variable de $Y$ (Buscar en alg�n estudio similar o suponer un valor de $CV_y=1$)\\
\end{document}